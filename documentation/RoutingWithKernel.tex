\include{head}

\begin{document}

\begin{figure}
\begin{center}
\begin{tikzpicture}[node distance=4mm, align=center, every path/.style={line}, every node/.style={font=\footnotesize}] %every join/.style={-latex'}
    \matrix[row sep=4mm,column sep=6mm] {
        \node [stop] (start) {start};\\
        \node [block] (zero) {\texttt{tempSurfaceStore}$=0;$\\ \texttt{discharge} $=0$};\\
        \node [block] (shuffle) {randomly shuffle positions};\\
        \node [block] (iterPrecip) {iteration over rainfall particles}; \\
        \node [block] (iterRaster) {iteration over $(x,y)$}; \\
        \node [block] (rnum) {create random number};\\
        \node [block] (saveStorage?) {is? \\ \texttt{rnum} $<$ infiltration Kernel at $(x,y)$ \\ and \\ store at $(x,y) <$ storage Kernel at $(x,y)$}; 
        & \node [block] (saveStorage) {save particle in storage};\\
        \node [decision] (boundary) {is \\ periodic boundary \\ condition?};\\
        \node [block] (lookupfdir) {lookup flow direction};\\
        \node [block] () {save particle in discharge};\\
        \node [block] (flowOverBoundary){flow reaches boundary?}; 
        & \node [block] (isDepression?) {is depression?}; 
        & \node [block] (tempSurfaceStore) {add particle to \\ temporal surface Storage};\\
        && \node [block] {spread out particle over \\ neighboring lower cells};\\
        \node [block] (outflow) {save particle in outflow};\\
        };
        \node [description, right=of shuffle] {with subroutine \texttt{TwoDRandPos()}};
        \node [description, right=of rnum] {\texttt{rnum}};

        \path [line] (saveStorage?) -- node [above] {yes} (saveStorage);
        \path [line] (saveStorage?) -- node [right] {no} (boundary);
        \path [line] (flowOverBoundary) -- node [right] {true} (outflow);
        \path [line] (flowOverBoundary) -- node [above] {else} (isDepression?);
        \path [line] (isDepression?) -- node [above] {true} (tempSurfaceStore);

\end{tikzpicture}
\end{center}
\caption{flow chart of subroutine \texttt{SimCode} (simulation execution)}
\end{figure}



\end{document}